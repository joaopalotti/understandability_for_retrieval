\begin{abstract}
In this paper we investigate methods to estimate the understandability of health Web pages, and use these to improve the retrieval of information for people seeking health advice on the Web.  Understandability plays a key role in ensuring that people accessing health information are capable of gaining insights that can assist them with their health concerns and choices. The access to unclear or misleading information has been shown to negatively impact on the health decisions of the general public. 

Our investigation considers methods to automatically estimate the understandability of health information in Web pages, and it provides a thorough evaluation of these methods using human assessments as well as an analysis of pre-processing factors affecting understandability estimations, and associated pitfalls. Furthermore, lessons learnt for estimating Web page understandability are applied to the construction of retrieval methods with specific attention to retrieving information understandable by the general public.

We found that machine learning techniques are more suitable to estimate health Web page understandability than traditional readability formulas and often used as guidelines and benchmarking by health information providers on the Web. Learning to rank effectively exploits these estimates to provide the general public with more understandable search results. These results are important for specialised search services tailored to support the general public in seeking health advice on the Web.


\end{abstract}


