
\section{Conclusion}
\label{sec:conclusion_doc_analysis}

\mytodo{Needs to be re-written}

There is an abundance of factors that affect how readability is perceived by users. 
In this paper we devised and studied a large number of readability estimators, ranging from traditional readability formulas extensively used in the past 50 years to state-of-the-art machine learning algorithms.
We grouped them into semantically related groups in order to visualize their correlation with human assessments collected during CLEF eHealth campaigns in 2015 and 2016.

Complementary to the literature research~\cite{palotti15}, we evaluated how preprocessing steps impact the readability estimation in traditional readability formulas and in other modern estimators. We empirically learnt the importance of preprocessing steps when applying readability formulas, as the highest correlations happen when other than the Naive method is used.
For the most modern estimators, such as the ones based on machine learning methods, the correlation is less sensible to the preprocessing steps.

We also studied the correlation of each individual readability formula to the human assessment to provide insights on which formula should be preferred. Our analysis concluded that the Simple Measure of Gobbledygook (SMOG) and Dale-Chall Index (DCI) were the most correlated metrics for the two datasets studied and, together with Coleman-Liau Index (CLI) and the Flesch Reading Ease (FRE) are the most stable metrics across datasets, and therefore, should be preferred.


