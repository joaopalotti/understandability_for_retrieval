\section{Integrating Understandability into Retrieval}
\label{sec:experiments}

% Modified this section to the simple present as we are talking about what we will do in the next section. Next section we use the simple past because we are reporting the experiments that we have already done. I think that is what makes more sense :P  (apart from the fact that changing from past to present we gain some space removing work endings)

%We investigate how understandability estimations could be used and integrated into retrieval methods to increase the quality of search results. We first investigate re-ranking search results from an initial run purely based on understandability estimations. 
We investigate how understandability estimations can be integrated into retrieval methods to increase the quality of search results. We start by evaluating re-ranking search results purely basing on understandability estimations. 

Specifically, we consider three retrieval methods of differing quality for the initial retrieval. These include the best two runs submitted to each CLEF task (2015 and 2016), and a plain BM25 baseline (default Terrier parameters: $b=0.75$ and $k1=1.2$). As understandability estimators we use eXtreme Gradient Boosting (XGB) regressor\footnote{For assessed documents, we used 10-fold cross validation, training XGB on 90\% of the data, and used its predictions for the remaining 10\%. For unassessed documents, we trained XGB on all assessed data, and applied this model to generate predictions. Different machine learning methods and feature selection schemes were experimented with; results are available in the online appendix. XGB was selected because its results were the best although other methods followed similar trends.}\cite{chen16}, as well as SMOG for CLEF 2015 and Dale-Chall for CLEF 2016. These were the best performing approaches from Section~\ref{sec:beyond_readability}.

If all the search results from a run were considered, then such a re-ranking method may place at early ranks Web pages highly likely to be understandable, but possibly less likely to be topically relevant. To balance relevance and understandability, we only re-rank the first $k$ documents. We explore $k = [15, 20, 50]$. Because evaluation is performed with respect to the first $n=10$ rank positions, the setting $k=15$ provided a conservative re-ranking of search results, while, $k=50$ provided a less conservative re-ranking approach. Results are presented in Section~\ref{results:reranking}.

As an alternative to the previous two-step ranking strategy for combining topical relevance and understandability, we explore the fusion of two search results lists separately obtained for relevance and understandability. For this, we use the Reciprocal Rank Fusion (RRF) method~\cite{cormack09}, which has proven effective for combining two lists of search results based on their document \textit{ranks}, rather than scores. This approach was selected above score-based fusion methods because of the different scoring strategies and distributions employed when scoring for relevance compared to for understandability. For topical relevance, we use, separately, the three methods used for re-ranking (ECNU~\cite{song15} and KISTI~\cite{oh15} for CLEF2015, GUIR~\cite{soldaini16} and ECNU~\cite{song16} for CLEF 2016, and BM25 for both collections). For understandability, we use, separately, the estimations from SMOG/Dale Chall and XGB. Also for this approach we study limiting the ranking of
results to be considered by the methods across the cut-offs $k=[15, 20, 50]$. Results are presented in Section~\ref{results:fusion}.

Finally, we consider a third alternative to combine topical relevance and understandability: using learning to rank with features derived from retrieval methods and the understandability estimators.
With the CLEF 2016 collection, we explore five combinations of label attribution and feature sets, keeping the same pairwise learning to rank algorithm based on tree boosting (XGB).
These combinations are listed in Table~\ref{tab:ltr}, with $R$ being the topical relevance of documents and $U$ their understandability assessments. 

%\vspace{-10pt}
\section{Evaluation of Understandability Retrieval}
\label{sec:results}

\begin{table}[!t]
\centering
\caption{Learning to rank settings. }
\vspace{-0.3cm}
\label{tab:ltr}
\resizebox{0.35\textwidth}{!}{
\begin{tabular}{l|l|l}
\toprule 
Name & Feature Set & Labeling Function\tabularnewline
\midrule 
Combo 1 & IR features & F(R,U) = $R$\tabularnewline
Combo 2 & IR + Unders. features & F(R,U) = $R$ \tabularnewline
Combo 3 & IR + Unders. features & F(R,U) = $R \times (100 - U)/100$ \tabularnewline
Combo 4 & IR + Unders. features & F(R,U) = $\begin{cases} R & \text{if } U \le 40\\ 0 & \text{otherwise}  \end{cases}$ \tabularnewline
Combo 5 & IR + Unders. features & F(R,U) = $\begin{cases} 2 \times R & \text{if } U \le 40\\ R & \text{otherwise}  \end{cases}$\tabularnewline
\bottomrule
\end{tabular}
}% end resizebox
\vspace{-14pt}
\end{table}


Results for the considered retrieval methods are reported in Table~\ref{tab:experiments}. We reported only the results for CLEF 2016 for brevity; those for CLEF 2015 exhibited similar trends and are included in the online appendix. The effectiveness of the top two submissions to CLEF 2016 and the BM25 baseline are reported at indices 1-3 of Table~\ref{tab:experiments}. Statistically significant differences compared to the best CLEF 2016 run, GUIR, are reported with $\diamond$; differences between an original run (indices 1-3) and its modifications are reported with $\dagger$ (paired, two-tail t-test, $p<0.05$). Note that the baseline BM25 is significantly worse than GUIR across all measures. 

%We start our experiments by showing at the top of Table~\ref{tab:experiments} (indices 1-3) the performance of the top 3 systems in CLEF eHealth 2016 together with a straightforward BM25 baseline run made with Terrier toolkit. Our further experiments will use not only these runs as a comparison base, but modify these runs when necessary.

\begin{table*}[ht!]
\caption{Results obtained by integrating understandability estimations within
retrieval methods on CLEF 2016. Baseline runs are reported at table
indices 1-3. Re-ranking experiments are reported at indices 4-21.
Fusion experiments are reported at indices 22-30. Learning to rank
experiments are reported at indices 31-35. All measures were calculated
up to rank $n=10$. }
%\vspace{-0.2cm}
 \label{tab:experiments} 
\resizebox{1.00\textwidth}{!}{ %
\begin{tabular}{cclllllllllllll}
\toprule 
    \multirow{2}{*}{Index }  & \multirow{2}{*}{Rerank }  & \multirow{2}{*}{Run }  & \multicolumn{4}{l}{Official CLEF 2016 Metrics} & \multicolumn{8}{l}{New Metrics to Evaluate Underst. in Retrieval - Sec.~\ref{sec:data}}\tabularnewline
\cmidrule(l{2pt}r{2pt}){4-7} \cmidrule(l{2pt}r{2pt}){8-15}  &  &  & $RBP$  & RBP Res.  & uRBP  & uRBP Res.  & $RBP_{u}$  & $RBP_{u}$ Res. & $HRBP$  & HRBP Res. & Unj  & $RBP_{r}^{*}$  & $RBP_{u}^{*}$  & $HRBP^{*}$\tabularnewline
\midrule 
1  & \multirow{3}{*}{No Rerank}  & GUIR~\cite{soldaini16} (Best Run)  & \textbf{28.11}  & 7.65  & \textbf{18.12}  & 7.19  & \textbf{45.69}  & 8.86 & \textbf{25.61}  & 6.50 & 0.01  & \textbf{28.29}  & \textbf{46.03}  & \textbf{25.79} \tabularnewline
2  &  & ECNU~\cite{song16} (Runner Up)  & 27.70  & 7.37  & 17.55  & \textbf{7.34}  & 43.89$^{\diamond}$  & 8.66 & 25.35  & 6.26 & 0.01  & 27.77  & 44.18$^{\diamond}$  & 25.48 \tabularnewline
3  &  & Plain BM25 Baseline  & 25.28$^{\diamond}$  & \textbf{8.24}  & 16.05$^{\diamond}$  & 6.94  & 42.08$^{\diamond}$  & \textbf{10.97} & 22.97$^{\diamond}$  & \textbf{7.19} & \textbf{0.06}  & {26.01}$^{\diamond}$  & {43.89}$^{\diamond}$  & {23.93}$^{\diamond}$ \tabularnewline
\midrule 
4  & \multirow{3}{*}{\makecell{Dale-Chall Top 15}}  & Based on GUIR  & 24.70$^{\dagger\diamond}$  & 8.70  & 16.83$^{\dagger\diamond}$  & 7.27  & 49.10$^{\dagger\diamond}$  & 10.62 & 24.94  & 7.50 & 0.03  & 25.24$^{\dagger\diamond}$  & 50.33$^{\dagger\diamond}$  & 25.54 \tabularnewline
5  &  & Based on ECNU  & 24.78$^{\dagger\diamond}$  & 7.83  & 16.64$^{\diamond}$  & 7.16  & 48.88$^{\dagger\diamond}$  & 9.71 & 24.80  & 6.50 & 0.02  & 25.12$^{\dagger\diamond}$  & 49.64$^{\dagger\diamond}$  & 25.21 \tabularnewline
6  &  & Based on BM25  & 23.22$^{\dagger\diamond}$  & 8.78  & 15.85 $^{\diamond}$  & 6.94  & 47.09$^{\dagger\diamond}$  & 11.83 & 24.01  & 7.42 & 0.07  & 24.04$^{\dagger\diamond}$  & 48.60$^{\dagger\diamond}$  & 24.82 \tabularnewline
\hdashline 7  & \multirow{3}{*}{\makecell{Dale-Chall Top 20}}  & Based on GUIR  & 22.19$^{\dagger\diamond}$  & 9.37  & 15.36$^{\dagger\diamond}$  & 6.98  & 48.71$^{\dagger\diamond}$  & 12.30 & 23.21$^{\dagger\diamond}$  & 8.12 & 0.06  & 23.26$^{\dagger\diamond}$  & 51.39$^{\dagger\diamond}$  & 24.45$^{\dagger\diamond}$\tabularnewline
8  &  & Based on ECNU  & 23.01$^{\dagger\diamond}$  & 8.93  & 15.70$^{\dagger\diamond}$  & 6.91  & 48.99$^{\dagger\diamond}$  & 11.69 & 23.73$^{\dagger\diamond}$  & 7.80 & 0.05  & 23.84$^{\dagger\diamond}$  & 51.00$^{\dagger\diamond}$  & 24.66\tabularnewline
9  &  & Based on BM25  & 21.58$^{\dagger\diamond}$  & 9.51  & 14.83$^{\dagger\diamond}$  & 7.02  & 46.99$^{\dagger}$  & 13.00 & 22.89$^{\diamond}$  & 8.06 & 0.09  & 22.93$^{\dagger\diamond}$  & 49.55$^{\dagger\diamond}$  & 24.26\tabularnewline
\hdashline 10  & \multirow{3}{*}{\makecell{Dale-Chall Top 50}}  & Based on GUIR  & 16.18$^{\dagger\diamond}$  & 15.24  & 11.56$^{\dagger\diamond}$  & 6.80  & 41.79$^{\dagger\diamond}$  & 24.49 & 18.10$^{\dagger\diamond}$  & 14.42 & 0.22  & 20.90$^{\dagger\diamond}$  & 53.28$^{\dagger\diamond}$  & 23.27$^{\dagger\diamond}$ \tabularnewline
11  &  & Based on ECNU  & 16.88$^{\dagger\diamond}$  & 17.37  & 11.78$^{\dagger\diamond}$  & \textbf{7.30}  & 40.76$^{\dagger\diamond}$  & 23.77 & 18.30$^{\dagger\diamond}$  & \textbf{15.57} & \textbf{0.24}  & 21.34$^{\dagger\diamond}$  & 52.07$^{\dagger\diamond}$  & 23.33$^{\dagger\diamond}$ \tabularnewline
12  &  & Based on BM25  & 15.06$^{\dagger\diamond}$  & 15.35$^{\dagger\diamond}$  & 10.55  & 6.62  & 40.03 $^{\diamond}$  & 23.88 & 16.55$^{\dagger\diamond}$  & 13.83 & \textbf{0.24}  & 19.42$^{\dagger\diamond}$  & 51.69$^{\dagger\diamond}$  & 21.59$^{\dagger\diamond}$ \tabularnewline
\hdashline 13  & \multirow{3}{*}{\makecell{XGB Top 15}}  & Based on GUIR  & \textbf{25.16}$^{\dagger\diamond}$  & 8.09  & \textbf{17.27}$^{\dagger\diamond}$  & 7.12  & \textbf{50.96}$^{\dagger\diamond}$  & 10.11 & \textbf{25.16}  & 6.89 & 0.02  & \textbf{25.61}$^{\dagger\diamond}$  & 52.00$^{\dagger\diamond}$  & \textbf{25.68}\tabularnewline
14  &  & Based on ECNU  & 24.18$^{\dagger\diamond}$  & 7.69  & 16.54 $^{\diamond}$  & 7.09  & 50.00$^{\dagger\diamond}$  & 9.91 & 24.56  & 6.65 & 0.02  & 24.56$^{\dagger\diamond}$  & 50.74$^{\dagger\diamond}$  & 25.01\tabularnewline
15  &  & Based on BM25  & 22.33$^{\dagger\diamond}$  & 8.14  & 15.46  & 6.76  & 47.90$^{\dagger\diamond}$  & 12.13 & 22.89$^{\diamond}$  & 7.25 & 0.07  & 23.11$^{\dagger\diamond}$  & 49.43$^{\dagger\diamond}$  & 23.69$^{\diamond}$\tabularnewline
\hdashline 16  & \multirow{3}{*}{\makecell{XGB Top 20}}  & Based on GUIR  & 22.38$^{\dagger\diamond}$  & 9.49  & 15.61$^{\dagger\diamond}$  & 7.05  & 50.45$^{\dagger\diamond}$  & 12.08 & 23.30$^{\dagger\diamond}$  & 8.16 & 0.05  & 23.62$^{\dagger\diamond}$  & 52.98$^{\dagger\diamond}$  & 24.68\tabularnewline
17  &  & Based on ECNU  & 22.95$^{\dagger\diamond}$  & 8.82  & 15.95$^{\dagger\diamond}$  & 7.02  & 50.42$^{\dagger\diamond}$  & 11.70 & 23.97$^{\diamond}$  & 7.56 & 0.04  & 23.68$^{\dagger\diamond}$  & 52.15$^{\dagger\diamond}$  & 24.73\tabularnewline
18  &  & Based on BM25  & 20.65$^{\dagger\diamond}$  & 9.42  & 14.46$^{\dagger\diamond}$  & 6.84  & 47.74$^{\dagger\diamond}$  & 13.56 & 21.93$^{\diamond}$  & 8.34 & 0.09  & 21.98$^{\dagger\diamond}$  & 50.28$^{\dagger\diamond}$  & 23.27$^{\diamond}$\tabularnewline
\hdashline 19  & \multirow{3}{*}{\makecell{XGB Top 50}}  & Based on GUIR  & 16.65$^{\dagger\diamond}$  & 15.73  & 12.39$^{\dagger\diamond}$  & 6.84  & 43.49$^{\dagger\diamond}$  & 23.63 & 18.70$^{\dagger\diamond}$  & 13.74 & 0.22  & 21.13$^{\dagger\diamond}$  & \textbf{55.07}$^{\dagger\diamond}$  & 23.58$^{\dagger\diamond}$\tabularnewline
20  &  & Based on ECNU  & 16.19$^{\dagger\diamond}$  & \textbf{17.01}  & 11.82$^{\dagger\diamond}$  & 7.27  & 43.05$^{\diamond}$  & \textbf{24.75} & 18.27$^{\dagger\diamond}$  & 14.41 & \textbf{0.24}  & 20.16$^{\dagger\diamond}$  & 54.70$^{\dagger\diamond}$  & 22.96$^{\dagger\diamond}$\tabularnewline
21  &  & Based on BM25  & 15.43$^{\dagger\diamond}$  & 15.37  & 11.33$^{\dagger\diamond}$  & 6.48  & 41.93$^{\diamond}$  & 23.65 & 17.43$^{\dagger\diamond}$  & 13.40 & 0.26  & 19.58$^{\dagger\diamond}$  & 54.04$^{\dagger\diamond}$  & 22.17$^{\dagger\diamond}$\tabularnewline
\midrule 
22  & \multirow{3}{*}{\makecell{RRF (XGB \& Orig.) Top 15} }  & Based on GUIR  & \textbf{27.23}$^{\dagger\diamond}$  & 7.76  & \textbf{18.31}  & \textbf{7.23}  & 49.69$^{\dagger\diamond}$  & 9.18 & 26.49$^{\dagger\diamond}$  & 6.62 & 0.01  & \textbf{27.46}$^{\dagger\diamond}$  & 50.07$^{\dagger\diamond}$  & \textbf{26.69}$^{\dagger\diamond}$\tabularnewline
23  &  & Based on ECNU  & 26.60$^{\dagger\diamond}$  & 7.41  & 17.81  & 7.19  & 48.67$^{\dagger\diamond}$  & 8.80 & 26.02  & 6.09 & 0.01  & 26.76$^{\dagger\diamond}$  & 49.10$^{\dagger\diamond}$  & 26.27$^{\dagger}$ \tabularnewline
24  &  & Based on BM25  & 24.57$^{\diamond}$  & 8.15  & 16.51$^{\diamond}$  & 6.91  & 46.76$^{\dagger}$  & 11.23 & 24.16$^{\dagger}$  & 7.20 & 0.06  & 25.32$^{\diamond}$  & 48.52$^{\dagger\diamond}$  & 25.08$^{\dagger}$ \tabularnewline
\hdashline 25  & \multirow{3}{*}{\makecell{RRF (XGB \& Orig.) Top 20}}  & Based on GUIR  & 26.21$^{\dagger\diamond}$  & 7.96  & 17.73  & 7.19  & 50.29$^{\dagger\diamond}$  & 9.58 & 25.89  & 6.73 & 0.03  & 26.53$^{\dagger\diamond}$  & 50.98$^{\dagger\diamond}$  & 26.25\tabularnewline
26  &  & Based on ECNU  & 26.15$^{\dagger\diamond}$  & 7.64  & 17.69  & 7.09  & 49.70$^{\dagger\diamond}$  & 9.28 & \textbf{26.07 } & 6.39 & 0.02  & 26.38$^{\dagger\diamond}$  & 50.32$^{\dagger\diamond}$  & 26.35\tabularnewline
27  &  & Based on BM25  & 24.04$^{\dagger\diamond}$  & 8.24  & 16.32$^{\diamond}$  & 6.87  & 47.69$^{\dagger\diamond}$  & 11.40 & 24.08$^{\dagger\diamond}$  & 7.35 & 0.06  & 24.82$^{\dagger\diamond}$  & 49.52$^{\dagger\diamond}$  & 25.01$^{\dagger}$ \tabularnewline
\hdashline 28  & \multirow{3}{*}{\makecell{RRF (XGB \& Orig.) Top 50}}  & Based on GUIR  & 24.09$^{\dagger\diamond}$  & \textbf{9.44}  & 16.85$^{\dagger\diamond}$  & 7.02  & 50.55$^{\dagger\diamond}$  & 11.76 & 24.76  & \textbf{8.01} & 0.07  & 25.08$^{\dagger\diamond}$  & \textbf{52.84}$^{\dagger\diamond}$  & 25.84\tabularnewline
29  &  & Based on ECNU  & 24.17$^{\dagger\diamond}$  & 8.67  & 16.75$^{\diamond}$  & 7.12  & \textbf{50.63}$^{\dagger\diamond}$  & 11.66 & 25.00  & 7.61 & 0.07  & 24.90$^{\dagger\diamond}$  & 52.50$^{\dagger\diamond}$  & 25.84 \tabularnewline
30  &  & Based on BM25  & 22.28$^{\dagger\diamond}$  & 8.87 & 15.50  & 6.76  & 48.79$^{\dagger\diamond}$  & \textbf{12.90} & 23.13$^{\dagger\diamond}$  & 7.82 & \textbf{0.10 } & 23.46$^{\dagger\diamond}$  & 51.89$^{\dagger\diamond}$  & 24.57\tabularnewline
\midrule 
31  & \multirow{5}{*}{XGB LeToR}  & Combo 1 on BM25  & 20.42$^{\dagger\diamond}$  & 17.61  & 13.00$^{\dagger\diamond}$  & 7.41  & 32.17$^{\dagger\diamond}$  & 24.61 & 18.39$^{\dagger\diamond}$  & 14.41 & 0.28  & 25.25$^{\diamond}$  & 43.19$^{\diamond}$  & 23.83$^{\diamond}$\tabularnewline
32  &  & Combo 2 on BM25  & 24.98$^{\dagger\diamond}$  & 19.83  & 15.30$^{\dagger\diamond}$  & 8.09  & 35.09$^{\dagger\diamond}$  & 25.14 & 22.26$^{\diamond}$  & 17.50 & 0.24  & 30.41  & 46.09  & 28.28$^{\dagger\diamond}$ \tabularnewline
33  &  & Combo 3 on BM25  & 26.35$^{\dagger}$  & \textbf{20.48}  & 15.88$^{\dagger\diamond}$  & 8.16  & 34.73$^{\dagger\diamond}$  & 24.69 & 21.81$^{\dagger}$  & 17.41 & 0.22  & 32.25$^{\diamond}$  & 45.44  & 28.22$^{\dagger\diamond}$\tabularnewline
34  &  & Combo 4 on BM25  & 16.16$^{\dagger\diamond}$  & 19.48  & 10.76$^{\dagger\diamond}$  & 7.27  & \textbf{36.75}$^{\dagger\diamond}$  & \textbf{28.51} & 16.77$^{\dagger\diamond}$  & \textbf{17.80} & \textbf{0.29 } & 22.20$^{\dagger\diamond}$  & \textbf{50.06}$^{\dagger\diamond}$  & 23.32$^{\diamond}$\tabularnewline
35  &  & Combo 5 on BM25  & \textbf{26.76}$^{\diamond}$  & \textbf{20.48}  & \textbf{16.19}$^{\diamond}$  & \textbf{8.34}  & 35.26$^{\dagger\diamond}$  & 24.13 & \textbf{22.96}  & 17.59 & 0.22  & \textbf{32.60}$^{\dagger}$  & 45.87  & \textbf{29.20}$^{\dagger\diamond}$\tabularnewline
\bottomrule
\end{tabular}
} % end of resizebox
\end{table*}


\subsection{Re-ranking}
\label{results:reranking}

Indices 4-12 of Table~\ref{tab:experiments} report the results of re-ranking methods applied to the runs listed at indices 1-3. Re-ranking was applied based on the DCI score of each document calculated using the preprocessing combination of Boilerpipe and ForcePeriod (best according to Pearson correlation, from Table~\ref{tab:top_corr_metrics}).
We found that the relevance of the re-ranked runs (as measured by $RBP_r$ and $RBP_r^*$) significantly decreased, compared to the original runs: e.g.,  re-ranking the top 15 search results using DCI  made $RBP_r$ decreasing from 25.28 to 21.58. However, these re-ranked results were significantly more understandable: for the previous example, $RBP_u$ passed from 42.08 to 47.09.
%Also note the limitation of uRBP metrics to reveal such a effect, as both relevance and understandability are tied together in one single score.

In the experiments, we also studied the influence of the numbers of documents considered for re-ranking (cut-off). Indices 4-6 refer to re-ranking only the top $n=15$ documents from the original runs; 7-9 refer to the first $n=20$; and 10-12 to the first $n=50$. The results show that the more documents are considered for re-ranking, the more degradation in $RBP_r$ effectiveness. Considering understandability-only in the evaluation, shows mixed results. Similar trends are observed for evaluation measures that consider understandability ($RBP$ and $RBP_u$), however with some exceptions. For example, an increase in $uRBP$ is observed when re-ranking ECNU using the top 50 results. 

Note that with the increase of the number of documents considered for re-ranking, there is an increase in number of unassessed documents being considered by the evaluation measures. Both the RBP residuals and the column \textit{Unj} quantify the effect unassessed documents have on evaluation. Nevertheless, we note that if unassessed documents are excluded from the evaluation, similar trends are observed, e.g. compare findings with those for $uRBP^*$, $RBP_r^*$, $RBP_u^*$ and $HRBP^*$.


Indices 13-21 refer to using the XGB regressor trained using all features listed in Table~\ref{tab:doc_features} to estimate understandability. Similarly to when using DCI, as the cut-off increases, e.g., from $n=15$ to $n=50$, documents returned are more understandable but less relevant. For the same cut-off value, e.g., $n=15$, the machine learning method used for estimating understandability consistently yielded more understandable results than DCI (higher $RBP_u$ and $RBP_u^*$). 

%Next, we swap the Dale-Chall Index by XGB regressor trained using all features listed in Table~\ref{tab:doc_features}. % When using the set of top 10 features for each group from Table~\ref{tab:top_corr_metrics} results are always better than using an automatic method such as Chi Square to select 10 features to use.}.
%While the higher correlation of Dale-Chall was XXX (as reported in table Y), correlations using XGB reached YYY, the closest method to the human agreement in this task.
%We experimented with different understandability-relevance trade-offs by re-ranking at cut-off 15 (a more conservative method), 20 and 50 (a less conservative method). 
%A similar understandability-relevance trade-off is seen when using a machine learning regressor in place of the Dale-Chall index.

\subsection{Rank Fusion}
\label{results:fusion}

Next, we report the results of automatically combining topical relevance and understandability through rank fusion (indices 22 to 30). We used the XGB method for estimating understandability, as it was the one yielding highest effectiveness for the re-ranking method. (Results for DCI are in the appendix and confirm the superiority of XGB.) 

Like as for re-ranking, also for rank fusion approaches we found that higher cut-offs were associated to higher effectiveness in terms of understandability measures on one hand, but higher losses in terms of relevance-oriented measures on the other. 
However, when combining relevance and understandability for evaluation ($HRBP$) is more stable now than for the previous re-rank methods. 
For example, compare the $HRBP^*$ of methods that re-rank the top 50 documents per topic, XGB (22.17) and RRF (24.57), with the original BM25 run (23.93).

\subsection{Learning to Rank}
\label{results:ltr}

Last, we evaluate the experiments with learning to rank, indices 31-35. We did not impose a maximum rank at which the learning-to-rank methods can re-rank, therefore we focus the following analysis on the metrics that ignore unassessed documents (the ones indicated with a $^*$).

Combo 1 at index 31 only used Information Retrieval features\footnote{We devised 24 IR features using the Terrier framework. The score of various retrieval models were extracted from a multi-field index composed of title, body and whole document.} and was trained only on topical relevance labels. Although simple, this is a typical learning to rank setting.
No significant difference was found between this setting and the original BM25 baseline for any metric.

Compared to Combo 1, Combo 2 (index 32) included the understandability features listed in Table~\ref{tab:doc_features}. Although not statistically different from the baseline, the inclusion of understandability features was as beneficial to the understandability relevance as for the topical relevance.

Combo 3 explored the use of understandability assessments of CLEF 2016 in a straightforward and effective manner. As the goal was to retrieve very easy-to-read documents (those with understandability label 0), a penalty was proportionally given to documents as far as their understandability score was further from the target. A document with understandability label 100 received label 0, while another with label of 50 had only half of its topical relevance score considered.

Combos 4 and 5 were devised based on a pre-defined understandability threshold $U$. While Combo 4 took into consideration only documents that are easy-to-read (understandability label $\le$ $U$), Combo 5 considered all documents, but boosted the relevance score of easy-to-read documents. In this work $U=40$, mimicking the evaluation formulas proposed in Section~\ref{sec:data}. While Combo 4 reached the highest understandability score for the learning-to-rank approaches ($RBP_u^{*}$ of 50.06), it failed to retrieve topically relevant documents ($RBP_r^{*}$ of only 22.20). In turn, Combo 5 reached the best understandability-relevance trade-off ($HRB^{*}$ of 29.20), being able to largely increase the topical relevance from the BM 25 baseline in which it was based ($RBP_r^*$ from 26.01 to 32.60 - an 25\% increase), while also improving understandability ($RBP_u^*$ from 43.89 to 45.87 - a 4\% increase). Note that Combo 5 is also significantly better than the best participant run of
CLEF 2016 for both $RBP_r^{*}$ (15\% increase) and $HRBP^{*}$ (13\% increase).


