
\section{A new Framework for Multidimensional IR Evaluation}
\label{sec:extension}

A limitation of UBIRE is that it prematurely combines the gains contributed by each dimension of relevance in \textbf{one} single step, providing a unique evaluation score~\cite{zuccon14,zuccon16}.
While this allows for the comparison of systems, it does not permit to understand the contribution each dimension had on the evaluation measure. 
To overcome this limitation, we aim to create a measure which, while still allowing the modelling of multidimensional relevance, is of easy interpretation and for which it is straightforward to track the contribution each relevance dimension had on the final effectiveness score. This is achieved by separating the evaluation of each dimension such that a value for each dimension is calculated separately with respect to its gain and discount, and then these are combined into a unique effectiveness measure. Note that we assume that it is possible to evaluate each measure separately: while this is akin to the compositionality assumption in UBIRE, if that failed, UBIRE would use mixture models to compute the related probabilities, while the proposed measure would be instead likely undefined. 

The evaluation of each relevance dimension separately is trivial, as it consists in applying the discount and gain function of the underlying evaluation measure, e.g. RBP, to each relevance dimension $\delta \in \mathcal{D}$, where the gains are those associated with the criteria for that specific dimension. 

While the outputs of each relevance dimension could be combined with a linear or geometric combination of values, we opt to use the weighted harmonic mean, as it is particularly sensitive to a single lower-than-average value. The same intuition is used to combine recall and precision in the widely used $F$-measure. 
Given a (discount-gain) evaluation measure $\mathcal{M}$, we apply the measure to evaluate a list of  documents $l_\delta$ which have been labeled with respect to dimension $\delta$ (i.e., we compute $\mathcal{M}(l_\delta)$). Then, to compute the proposed measure $MM_\mathcal{M}$, we combine all $\mathcal{M}(l_\delta)$ for each relevance dimension using the harmonic mean, where each dimension is weighted according to a preferential weight $w_\delta$ assigned to each dimension; formally:

\vspace{-4pt}
\begin{equation}
    MM_\mathcal{M}  = \left( \frac{\sum\limits_{\delta=1}^n w_\delta \cdot \mathcal{M}(l_\delta)^{-1}}{\sum\limits_{\delta=1}^n w_\delta} \right)^{-1}
          = \frac{\sum\limits_{\delta=1}^n w_\delta}{\sum\limits_{\delta=1}^n \frac{w_\delta}{\mathcal{M}(l_\delta)}}
\label{eq:MM}
\end{equation}

%
Without loss of generality, we instantiate $\mathcal{M} = RBP$ and define the following modification of RBP~\cite{moffat08} for each dimension:
%
\begin{itemize}[leftmargin=*]
	\item $RBP_t(\rho)$: uses binary topicality assessments (i.e. the usual RBP). 
%	
    \item $RBP_u(\rho)$: uses understandability assessments (either graded or binary; see below for specific instantiations). 
\end{itemize}

\noindent Thus Equation~\ref{eq:MM} becomes (we assumed $w_t = w_u$):
%
\begin{equation}
 MM_{RBP(\rho)} = 2 \cdot \frac{RBP_t(\rho) \cdot RBP_u(\rho)}{RBP_t(\rho) + RBP_u(\rho)}
\label{eq:MMrbp}
\end{equation}



%%%%%%%%%%%%%%%%%%%%%%%%%%%%%%%%%%%%%%%%%%%%%%%%%%%%%%%%%%%%%%%%%%%%%%%%%%%%%%%%%%%%%%%%%%%%%%%%%%%%%%%%%%%%%%%%%%%%%%%%%%%%%%%%
%%%%%%%%%%%%%%%%%%%%%%%%%%%%%%%%%%%%%%%%%%%%%%%%%%%%%%%%%%%%%%%%%%%%%%%%%%%%%%%%%%%%%%%%%%%%%%%%%%%%%%%%%%%%%%%%%%%%%%%%%%%%%%%%
%%%%%%%%%%%%%%%%%%%%%%%%%%%%%%%%%%%%%%%%%%%%%%%%%%%%%%%%%%%%%%%%%%%%%%%%%%%%%%%%%%%%%%%%%%%%%%%%%%%%%%%%%%%%%%%%%%%%%%%%%%%%%%%%
%%%%%%%%%%%%%%%%%%%%%%%%%%%%%%%%%%%%%%%%%%%%%%%%%%%%%%%%%%%%%%%%%%%%%%%%%%%%%%%%%%%%%%%%%%%%%%%%%%%%%%%%%%%%%%%%%%%%%%%%%%%%%%%%

