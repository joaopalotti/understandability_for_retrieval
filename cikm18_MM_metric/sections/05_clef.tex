
\begin{table}[t]
    \centering
    \caption{Kendall-$\tau$ correlation for systems participating in CLEF eHealth 2015 and 2016.     \vspace{-6pt}
    %uRBP merges the understandability labels with the default RBP using UBIRE framework. $H_{RBP}$ is the weighted geometric mean of RBP and $RBP_u$: the former measures only topicability and the latter, only understandability.
    }
    \label{tab:kendall}
    %
    \resizebox{0.45\textwidth}{!}{ %
    \begin{tabular}{@{}l|llll|llll@{}}
        \toprule
                  & \multicolumn{4}{c}{\textbf{CLEF 2015}} & \multicolumn{4}{c}{\textbf{CLEF 2016}} \\ \cmidrule{2-9} 
                  & RBP     & uRBP    & $RBP_u$   & $H_{RBP}$  & RBP     & uRBP    & $RBP_u$   & $H_{RBP}$  \\ 
                  RBP           & 1.000   & 0.901   & 0.483    & 0.843   & 1.000   & 0.948   & 0.497    & 0.850   \\
                  uRBP          & 0.901   & 1.000   & 0.563    & 0.901   & 0.948   & 1.000   & 0.456    & 0.866   \\
                  $RBP_u$       & 0.483   & 0.563   & 1.000    & 0.610   & 0.497   & 0.524   & 1.000    & 0.633   \\
                  $H_{RBP}$     & 0.843   & 0.901   & 0.610    & 1.000   & 0.850   & 0.866   & 0.633    & 1.000   \\ 
        \bottomrule
    \end{tabular}
    } % end resizebox
    \vspace{-18pt}
\end{table}




\section{Rank Correlations} %across Systems and Measures}
\label{sec:clef}
Next, we compared the behaviours of $MM$ and UBIRE using real data. For this, we used the systems participating to the CLEF eHealth IR Lab evaluations in 2015 and 2016~\cite{clefIR15,clefIR16}. In both these evaluation challenges, systems were officially evaluated using $uRBP$ -- we further evaluated each system using $MM$ and studied the correlations among system rankings obtained using RBP (thus considering topicality only), $uRBP$ (UBIRE), and our proposed $RBP_u$ (thus considering only understandability) and $MM_{RBP}$. This investigation of correlations is a common approach to compare and understand relative behaviour of evaluation measures~\cite{zuccon16}. 

Specifically, we studied a setting where understandability was binary, akin to topicality, which also was considered as binary. For topicality, this was achieved using the common gain function for RBP that only models binary relevance: graded relevance labels were conflated to binary such that highly relevant and relevant assessments were mapped to relevant, and the rest to irrelevant. For understandability, the binarisation of the assessments was dependant on the year of the challenge. For 2015, understandability assessments were made on a 4-point scale (\textit{very easy}, \textit{easy}, \textit{hard} and \textit{very hard})~\cite{clefIR15}: we made this binary by assuming that a document was understandable if assessed as \textit{very easy} or \textit{easy}, and not-understandable otherwise. For 2016, understandability assessments were made on an integer scale ranging from 0 (very easy) to 100 (very hard)~\cite{clefIR16}: we made this binary by assuming that documents with an assessment lower than or equal to 40 were understandable, while we made the remaining as not-understandable. 

Table~\ref{tab:kendall} shows the Kendall-$\tau$ rank correlations of systems according to RBP, $uRBP$, $RBP_u$ and $MM_{RBP}$. Rank correlation between RBP and $uRBP$ was high for both 2015 and 2016 data. This emphasises the tight relation between RBP and $uRBP$. On the other hand, $MM_{RBP}$ exhibited the strongest rank correlation with $RBP_u$, while the correlation between $RBP_u$ and RBP or $uRBP$ is marginal. In addition, we found that $MM_{RBP}$ strongly correlated with RBP, but not as strongly as  $uRBP$ does. Finally, $MM_{RBP}$ and $uRBP$ showed a generally high correlation among themselves, highlighting that the two measures provided similar evaluations of system effectiveness; however, $MM_{RBP}$ had the advantage that the trade-off between topicality and understandability could be clearly identified and studied. 

