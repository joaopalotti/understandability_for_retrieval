\begin{abstract}

%Relevance is multidimensional. But evaluation campaigns are still measuring it as if it were purely based on topicality. 
%One reason for that might be the lack of evaluation metrics/frameworks that incorporate another dimension other than topicality.
In this paper we proposed a framework to evaluate information retrieval systems in presence of multidimensional relevance. This is an important problem in tasks such as consumer health search, where the understandability and trustworthiness of information greatly influence people's decisions based on the search engine results. 
We used synthetic and real data to compare our proposed framework, named $H$, to the understandability-biased information evaluation (UBIRE), an existing framework used in the context of consumer health search. We showed how the proposed approach diverges from the UBIRE framework, and how $H$ can be used to better understand the trade-offs between topical relevance and the other relevance dimensions. 






%%%%With two experiments, we compare our framework, named $H$, to UBIRE, an existing framework used in the context of consumer health search
%%%Through experiments with synthetic systems and a correlation analysis, we compare our framework, named $H$, to UBIRE, an existing framework used in the context of consumer health search.
%%%%
%%%%We propose two experiments to compare our framework, named $H$, to UBIRE, an existing framework used in the context of consumer health search.
%%%%Firstly, we employ synthetic systems, which allows having fine-grained control, to understand the advantages of H over UBIRE.
%%%%Secondly, we study the rank correlations of systems evaluated with metrics in $H$ and UBIRE.
%%%%
%%%We conclude that while system correlations measured by metrics in $H$ are not discrepant from the ones reached by UBIRE,
%%%$H$ provides more information to researchers/experimenters, allowing them to easily assess and control how each relevance dimension contributes to the final score of a system.

\end{abstract}

