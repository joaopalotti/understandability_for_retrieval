\IEEEraisesectionheading{\section{Introduction} \label{chp:understanding_understandability}}

\IEEEPARstart{S}{earch} engines are concerned with retrieving relevant information to support a user's information seeking task. Commonly, signals about the topicality or aboutness of a piece of information with respect to a query are used to estimate relevance, with other relevance dimensions like understandability, trustworthiness, etc.~\cite{zhang2014multidimensional} being relegated to a secondary position, or completely neglected. While this may be a minor problem for many information seeking tasks, there are some specific tasks in which dimensions other than topicality have an important role in the information seeking and decision making process. The seeking of health information and advice on the Web by the general public is one such task. 

A key problem when searching the Web for health information is that this can be too technical, unreliable, generally misleading, and can lead to unfounded escalations and poor decisions~\cite{white09b}. Where correct information exists, it can be hard to find and digest amongst the noise, spam, technicalities, and irrelevant information. In \textit{high-stakes search tasks} such as this, access to poor information can lead to poor decisions which ultimately can have a significant impact on our health and well-being~\cite{white09b,white13}. In this work we are specifically interested in the understandability of health information retrieved by search engines, and in improving search results to favour information understandable by the general public. 

The use of general purpose Web search engines like Google, Bing and Baidu for seeking health advice has been largely analysed, questioned and criticised~\cite{graber99,fitzsimmons10,wiener13,patel13,atcherson14,meillier17,ellimoottil12}, despite the commendable efforts these services have put into providing increasingly better health information, e.g., the Google Health Cards~\cite{gabrilovich2016cura}. 

Ad-hoc solutions to support the general public in searching and accessing health information on the Web have been implemented, typically supported by government initiatives or medical practitioner associations, e.g., \url{HealthOnNet.org} (HON) and \url{HealthDirect.gov.au}, among others. These solutions aim to provide \textit{better} health information to the general public. For example, HON's mission statement is ``to guide Internet users to reliable, understandable, accessible and trustworthy sources of medical and health information''. But, do the solutions these services currently employ actually provide this type of information to the health-seeking general public? As an illustrative example, we analysed the top 10 search results retrieved by HON\footnote{Results retrieved on 01/10/2017.} in answer to 300 search queries from CLEF 2016 eHealth (see Section~\ref{sec:data}). Figure~\ref{fig:dist} reports the cumulative distribution of understandability scores for these search results (note, we did not assess their topical relevance). Understandability scores were computed with the most effective readability formula and settings from Section~\ref{sec:which_preprocessing} (Dale-Chall Index), and express how easy is to understand a Web page. Low scores correspond to easy to understand Web pages. We report also the scores for the ``optimal'' search results (Oracle), as found in CLEF 2016 (relevant results that have the highest understandability scores), along with the scores for the best retrieval method from Section~\ref{sec:results}. The results clearly indicate that, despite solutions like HON being explicitly aimed at supporting access to understandable health information, they often fail to do so.

In this paper we proposed and investigated methods for the estimation of the understandability of health information in Web pages. In doing so, we also studied the influence of HTML processing methods on these estimations, and their pitfalls. Then, we investigated how understandability estimations can be integrated into retrieval methods to enhance the quality of the retrieved health information, with particular attention to its understandability by the general public. This paper makes a concrete contribution to practice, as it informs health search engines specifically tailored to the general public about the best methods they should adopt. 

\begin{figure}[t!]
   \centering
    %\vspace{-0.5cm}
   \includegraphics[width=.41\textwidth]{graphics/cumdist}
    %\vspace{-0.2cm}
    \caption{Distribution of Dale-Chall Index (DCI) of search results. DCI measures the years of schooling required to understand a document. The average US resident reads at or below an 8th grade level (dashed line)\cite{cowan04,wallace04,davis04,stossel12}, which is the level suggested by NIH for health information on the Web~\cite{clear94}. The distribution for HON is similar to that of the baseline used in this paper (BM25). Our best method (XGB) re-ranks documents to provide more understandable results; its distribution is similar to that of an ``Oracle'' system.}
   \label{fig:dist}
\end{figure}


