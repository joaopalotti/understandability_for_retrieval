
%\vspace{-10pt}
\subsection*{Evaluation of Understandability Retrieval}
\label{sec:results}


\begin{table}
\caption{Reports on the experiments with 4 base runs: the top 3 runs of CLEF eHealth 2017 and a plain baseline run. The second (indices 5-8) and third (indices 9-12) parts shows results of a RegressorTree based on top features from Table X. The forth part of this table shows results when reranking top 20 results based on Dale-Chall Index. Finally, we combine selected runs with reciprocal rank fusion in the last part of this table. Results shows that regression on top 15 improves
    understandability to the detriment of topical relevance. Regression on top 20 it is even more agreessive. Dale Chall presents the same logic, but understandability gains are smaller compared to regressor results. Finally, the combination with rrf yields the best $F_ru$ scores. I still need to compute if we get statistically significant improvements or not.}
\label{tab:experiments}
\begin{tabular}{c|c|c|ccc|cc|c|ccc}
\toprule 
Index & Rerank & Run & P\_r@10 & P\_u@10 & F\_ru & RBP & uRBP & Unj@10 & Pr@10 & Pu@10 & F\_ru{*}\tabularnewline
\midrule 
1 & \multirow{4}{*}{No Rerank} & BM25 Q.E. & 31.43 & 49.70 & 38.51 & 29.05 & 10.40 & 0.02 & 31.70 & 50.83 & 39.05\tabularnewline
2 &  & GUIR & 29.67 & 50.97 & 37.50 & 28.11 & 9.98 & 0.01 & 30.20 & 51.80 & 38.16\tabularnewline
3 &  & ECNU & 29.33 & 49.27 & 36.77 & 27.70 & 10.14 & 0.01 & 29.50 & 49.87 & 37.07\tabularnewline
4 &  & Plain BM25 & 26.47 & 46.33 & 33.69 & 25.28 & 9.22 & 0.06 & 27.63 & 48.93 & 35.32\tabularnewline
\midrule
5 & \multirow{4}{*}{Regress Top 15} & Based on Run 1 & 27.87 & 49.40 & 35.63 & 26.38 & 6.51 & 0.01 & 28.10 & 49.60 & 35.88\tabularnewline
6 &  & Based on Run 2 & 27.27 & 48.57 & 34.93 & 25.95 & 6.56 & 0.01 & 27.30 & 48.60 & 34.96\tabularnewline
7 &  & Based on Run 3  & 26.60 & 50.10 & 34.75 & 25.08 & 6.23 & 0.00 & 26.67 & 50.10 & 34.81\tabularnewline
8 &  & Based on Run 4 & 23.73 & 50.87 & 32.37 & 22.86 & 5.19 & 0.09 & 24.97 & 52.67 & 33.87\tabularnewline
\midrule 
9 & \multirow{4}{*}{Regress. Top 20} & Based on Run 1 & 27.70 & \textbf{52.13} & 36.18 & 26.10 & 5.95 & 0.02 & 28.17 & 52.53 & 36.67\tabularnewline
10 &  & Based on Run 2 & 26.70 & \textbf{53.10} & 35.53 & 25.44 & 5.81 & 0.01 & 26.83 & 53.33 & 35.70\tabularnewline
11 &  & Based on Run 3  & 26.17 & \textbf{52.67} & 34.96 & 24.88 & 5.59 & 0.02 & 26.43 & 53.17 & 35.31\tabularnewline
12 &  & Based on Run 4 & 23.30 & \textbf{51.43} & 32.07 & 22.53 & 4.51 & 0.12 & 25.10 & 55.20 & 34.51\tabularnewline
\midrule 
13 & \multirow{4}{*}{Dale-Chall top 20} & Based on Run 1 & 29.63 & 49.93 & 37.19 & 26.45 & 8.58 & 0.02 & 29.87 & 50.37 & 37.50\tabularnewline
14 &  & Based on Run 2 & 28.37 & 51.10 & 36.48 & 25.01 & 8.05 & 0.01 & 28.60 & 51.40 & 36.75\tabularnewline
15 &  & Based on Run 3  & 27.70 & 50.80 & 35.85 & 24.54 & 7.96 & 0.02 & 28.03 & 51.47 & 36.30\tabularnewline
16 &  & Based on Run 4 & 24.13 & 49.97 & 32.55 & 21.28 & 6.68 & 0.10 & 26.17 & 53.17 & 35.07\tabularnewline
\midrule
17 & \multirow{4}{*}{Reciprocal Rank Fusion} & Runs 1 and 9 & 30.70 & 52.37 & \textbf{38.71} & 27.93 & 7.63 & 0.02 & 31.10 & 53.30 & 39.28\tabularnewline
18 &  & Runs 2 and 10 & 30.67 & 51.90 & \textbf{38.55} & 28.34 & 8.14 & 0.01 & 30.97 & 52.53 & 38.96\tabularnewline
19 &  & Runs 3 and 11 & 30.40 & 51.37 & \textbf{38.20} & 27.82 & 8.00 & 0.01 & 30.67 & 51.90 & 38.55\tabularnewline
20 &  & Runs 4 and 12 & 28.67 & 50.67 & \textbf{36.62} & 26.33 & 7.56 & 0.03 & 29.37 & 51.87 & 37.50\tabularnewline
\bottomrule
\end{tabular}
\end{table}


Results for the considered retrieval methods are reported in Table~\ref{tab:experiments}. We report only the results for CLEF 2016 for brevity; those for CLEF 2015 exhibited similar trends and are included in the online appendix. The effectiveness of the top two submissions to CLEF 2016 and the BM25 baseline are reported at indices 1-3 of Table~\ref{tab:experiments}. Statistically significant differences compared to the best CLEF 2016 run, GUIR, are indicated with $\diamond$; differences between an original run (indices 1-3) and its modifications are indicated with $\dagger$ (paired, two-tail t-test, $p<0.05$). Note that the baseline BM25 is significantly worse than GUIR across all measures. 

%We start our experiments by showing at the top of Table~\ref{tab:experiments} (indices 1-3) the performance of the top 3 systems in CLEF eHealth 2016 together with a straightforward BM25 baseline run made with Terrier toolkit. Our further experiments will use not only these runs as a comparison base, but modify these runs when necessary.

%\vspace{-6pt}
%\subsubsection{Re-ranking}
%\label{results:reranking}

Indices 4-12 of Table~\ref{tab:experiments} report the results of re-ranking methods applied to the runs listed at indices 1-3. Re-ranking was applied based on the DCI score of each document calculated using the preprocessing combination of Boilerpipe and ForcePeriod (best according to Pearson correlation, from Table~\ref{tab:top_corr_metrics}).
We found that the relevance of the re-ranked runs (as measured by $RBP_r$ and $RBP_r^*$) significantly decreased, compared to the original runs: e.g., re-ranking the top 15 search results using DCI  made $RBP_r$ decreasing from 25.28 to 21.58. However, these re-ranked results were significantly more understandable: for the previous example, $RBP_u$ passed from 42.08 to 47.09.
%Also note the limitation of uRBP metrics to reveal such a effect, as both relevance and understandability are tied together in one single score.

In the experiments, we also studied the influence of the numbers of documents considered for re-ranking (cut-off). Indices 4-6 refer to re-ranking only the top $k=15$ documents from the original runs; 7-9 refer to the first $k=20$; and 10-12 to the first $k=50$. The results show that the more documents are considered for re-ranking, the more degradation in $RBP_r$ effectiveness. Considering understandability-only in the evaluation shows mixed results. Similar trends were observed for evaluation measures that consider understandability ($RBP$ and $RBP_u$), however with some exceptions. For example, an increase in $uRBP$ was observed when re-ranking ECNU using the top 50 results. 

Note that with the increase of the number of documents considered for re-ranking, there is an increase in number of unassessed documents being considered by the evaluation measures. Both the RBP residuals and the column \textit{Unj} quantify the effect unassessed documents have on evaluation. Nevertheless, we note that if unassessed documents are excluded from the evaluation, similar trends are observed, e.g., compare findings with those for $uRBP^*$, $RBP_r^*$, $RBP_u^*$ and $H_{RBP}^*$.


Indices 13-21 refer to using the XGB regressor trained with all features listed in Table~\ref{tab:doc_features} to estimate understandability. Similarly to when using DCI, as the cut-off increased, e.g., from $k=15$ to $k=50$, the documents returned were more understandable but less relevant. For the same cut-off value, e.g., $k=15$, the machine learning method used for estimating understandability consistently yielded more understandable results than DCI (higher $RBP_u$ and $RBP_u^*$). 

Overall, statistical significant improvements over the baselines were observed for most configurations and measures.  

%Next, we swap the Dale-Chall Index by XGB regressor trained using all features listed in Table~\ref{tab:doc_features}. % When using the set of top 10 features for each group from Table~\ref{tab:top_corr_metrics} results are always better than using an automatic method such as Chi Square to select 10 features to use.}.
%While the higher correlation of Dale-Chall was XXX (as reported in table Y), correlations using XGB reached YYY, the closest method to the human agreement in this task.
%We experimented with different understandability-relevance trade-offs by re-ranking at cut-off 15 (a more conservative method), 20 and 50 (a less conservative method). 
%A similar understandability-relevance trade-off is seen when using a machine learning regressor in place of the Dale-Chall index.

%
%\vspace{-6pt}
%\subsection{Rank Fusion}
%\label{results:fusion}
%\vspace{-2pt}

Next, we report the results of automatically combining topical relevance and understandability through rank fusion (indices 22 to 30). We used the XGB method for estimating understandability, as it was the one yielding highest effectiveness for the re-ranking method. Runs were thus produced by fusing the re-ranking with XGB and the original run. (Results for DCI are reported in the online appendix and confirm the superiority of XGB.) 

Like as for re-ranking, also for the rank fusion approaches we found that, in general, higher cut-offs were associated to higher effectiveness in terms of understandability measures on one hand, but higher losses in terms of relevance-oriented measures on the other. Overall, results obtained with rank fusion were superior to those obtained with re-ranking only, though most differences were not statistically significant. Statistical significant improvements over the baselines were instead observed for most configurations and measures.  
%However, when combining relevance and understandability for evaluation ($H_{RBP}$) is more stable now than for the previous re-rank methods. 
%For example, compare the $H_{RBP}^*$ of methods that re-rank the top 50 documents per topic, XGB (22.17) and RRF (24.57), with the original BM25 run (23.93).

%\vspace{-6pt}
%\subsection{Learning to Rank}
%\label{results:ltr}
%\vspace{-2pt}

Last, we analyse the results obtained by the learning to rank methods (indices 31-35). Unlike with the previous methods, we did not impose a rank cut-off on learning to rank. Learning to rank was only applied to the BM25 baseline, as we had no access to the IR features for the runs submitted at CLEF (i.e. GUIR and ECNU for CLEF 2016).

When considering $RBP_r$ and $uRBP$, learning to rank exhibited effectiveness that was significantly inferior to that of the GUIR and ECNU baseline runs, though higher than those for the BM25 baseline (for some configurations). The examination of the RBP residuals (and the number of unassessed documents) revealed that this may have been because measures were affected by the large number of unassessed documents retrieved in the top 10 ranks. For example, the $RBP_r$ residual for learning to rank methods was about double than that of the baselines or other approaches. In fact, among the documents retrieved in the top 10 results by learning to rank, there were 20\% that were unassessed, compared to an average of 3\% for the other methods. (Excluding XGB with cut-off 50, which also exhibited high residuals). 

%We did not impose a maximum rank at which the learning-to-rank methods can re-rank, therefore we focus the following analysis on the metrics that ignore unassessed documents (the ones indicated with a $^*$).

We thus should carefully account for unassessed documents through considering the residuals of RBP measures, as well as the measures that ignore unassessed documents. When this was done, we observed that learning to rank methods overall provided substantial gains over the original runs and other methods (when considering $RBP^*_r$, $RBP^*_u$ and $H_{RBP}^*$), or large potential gains over these methods (when considering the residuals). Next, we analyse these results in more detail. 

No improvements over the baselines were found for Combo 1 (index 31), and the high residuals for $RBP_r$ were not matched by other residuals or by considering only assessed documents. Combo 1 was a simple method that used only IR features and was trained only on topical relevance. Specifically, we devised 24 IR features using the Terrier framework. The score of various retrieval models was extracted from a multi-field index composed of title, body and whole document. Although simple, this is a typical learning to rank setting.

Compared to Combo 1, Combo 2 (index 32) included the understandability features listed in Table~\ref{tab:doc_features}. This inclusion was as beneficial to the understandability measures as to the relevance measures, with $RBP_r^*$, $RBP_u^*$ and $H_{RBP}^*$ all showing gains over the baselines. Combo 3 obtained similar $H_{RBP}^*$ values, though with higher effectiveness for relevance measures ($RBP_r^*$) than for understandability ($RBP_u^*$).

%Combo 3 explored the use of understandability assessments of CLEF 2016 in a straightforward and effective manner. As the goal was to retrieve very easy-to-read documents (those with understandability label 0), a penalty was proportionally given to documents as far as their understandability score was further from the target. A document with understandability label 100 received label 0, while another with label of 50 had only half of its topical relevance score considered.

Combos 4 and 5 were devised based on a set understandability threshold $U=40$. While Combo 4 took into consideration only documents that are easy-to-read (understandability label $\le$ $U$), Combo 5 considered all documents, but boosted the relevance score. Combo 4 reached the highest understandability score for the learning-to-rank approaches ($RBP_u^{*}=50.06$), but it failed to retrieve a substantial number of relevant documents ($RBP_r^{*}=22.20$). In turn, Combo 5 reached the highest understandability-relevance trade off ($HRB^{*}=29.20$). Compared to the BM25 baseline (on which it was based), Combo 5  largely increased both relevance ($RBP_r^*$ from 26.01 to 32.60 -- a 25\% increase) and understandability ($RBP_u^*$ from 43.89 to 45.87 -- a 4\% increase). Note that Combo 5 was also significantly better than the best run submitted to CLEF 2016 for both $RBP_r^{*}$ (15\% increase) and $H_{RBP}^{*}$ (13\% increase).


