\section*{Evaluation of understandability estimators}
\label{sec:beyond_readability}


\begin{table}[t]
\centering    
\caption{Metrics with highest correlation per group. In bold are the metric that archived the highest correlation for a correlation measure.}
\label{tab:top_corr_metrics}
\resizebox{.45\textwidth}{!}{ %%%%
\begin{tabular}{c|c|c|c|c|c|c}
\toprule
\textbf{Dataset} & \textbf{Group} & \textbf{Metric} & \textbf{Preproc.} & \textbf{Pears.} & \textbf{Spear.} & \textbf{Kend.}\tabularnewline
\midrule
\multirow{15}{*}{CLEF 2015} & RF & SMOG Index & JST NFP & \textbf{0.438} & \textbf{0.388} & \textbf{0.286}\tabularnewline
\cmidrule{2-7} 
 & \multirow{2}{*}{CRF} & Avg. Num. of Polysyl. Words per Word & JST FP & \textbf{0.429} & 0.364 & 0.268\tabularnewline
 &  & Avg. N. of Polysyl. Words per Sentence & JST NFP & 0.192 & \textbf{0.388} & \textbf{0.286}\tabularnewline
\cmidrule{2-7} 
& \multirow{2}{*}{GMV} & Avg. N. Medical Prefixes per Word & Naive FP & \textbf{0.314} & 0.312 & 0.229\tabularnewline
 &  & Number of Medical Prefixes & Naive FP & 0.131 & \textbf{0.368} & \textbf{0.272}\tabularnewline
\cmidrule{2-7} 
 & CMV & CHV Mean Score for all Concepts & Naive FP & \textbf{0.371} & \textbf{0.314} & \textbf{0.228}\tabularnewline
\cmidrule{2-7} 
 & EMV & Number of MeSH Concepts & Naive FP & \textbf{0.227} & \textbf{0.249} & \textbf{0.178}\tabularnewline
\cmidrule{2-7} 
 &  \multirow{2}{*}{NLF} & N. of words not found in Aspell Dict. & JST NFP & \textbf{0.351} & 0.276 & 0.203\tabularnewline
 &  & Number of Pronouns per Word & Naive FP & 0.271 & \textbf{0.441} & \textbf{0.325}\tabularnewline
\cmidrule{2-7} 
 & HF & Number of P Tags & None & \textbf{0.219} & \textbf{0.196} & \textbf{0.142}\tabularnewline
\cmidrule{2-7} 
 &  \multirow{2}{*}{WFF} & Mean Rank Medical Reddit - Includes OV & JST NFP & \textbf{0.435} & 0.277 & 0.197\tabularnewline
 &  & 25th percentil Pubmed & JST NFP & 0.330 & \textbf{0.347} & \textbf{0.256}\tabularnewline
\cmidrule{2-7} 
 &  \multirow{2}{*}{MLR} & Neural Network Regressor & BOI NFP & \textbf{0.602} & 0.394 & 0.287\tabularnewline
 &  & Neural Network Regressor & JST FP & 0.565 & \textbf{0.438} & \textbf{0.324}\tabularnewline
\cmidrule{2-7} 
 & MLC & Multinomial Naive Bayes & Naive FP & \textbf{0.573} & \textbf{0.477} & \textbf{0.416}\tabularnewline
\midrule
\midrule
\multirow{18}{*}{CLEF 2016} & \multirow{2}{*}{RF} & Dale Chall Index & JST FP & \textbf{0.439} & 0.381 & 0.264\tabularnewline
 &  & Dale Chall Index & BOI FP & 0.437 & \textbf{0.382} & \textbf{0.264}\tabularnewline
\cmidrule{2-7} 
 & CRF & Avg. Difficult Words Per Word & BOI FP & \textbf{0.431} & \textbf{0.379} & \textbf{0.262}\tabularnewline
\cmidrule{2-7} 
 & \multirow{2}{*}{GMV} & Avg. Prefixes per Sentence & JST FP & \textbf{0.263} & 0.242 & 0.164\tabularnewline
 &  & ICD Concepts Per Sentence & JST NFP & 0.014 & \textbf{0.253} & \textbf{0.172}\tabularnewline
\cmidrule{2-7} 
 & \multirow{2}{*}{CMV} & CHV Mean Score for all Concepts & JST FP & \textbf{0.329} & 0.313 & 0.216\tabularnewline
 &  & CHV Mean Score for all Concepts & BOI FP & 0.329 & \textbf{0.325} & \textbf{0.224}\tabularnewline
\cmidrule{2-7} 
 & \multirow{2}{*}{EMV} & Number of MeSH Concepts & BOI NFP & \textbf{0.201} & 0.166 & 0.113\tabularnewline
 &  & Number of MeSH Disease Concepts & BOI NFP & 0.179 & \textbf{0.192} & \textbf{0.132}\tabularnewline
\cmidrule{2-7} 
 & \multirow{2}{*}{NLF} & Avg. Stopword Per Word & BOI FP & \textbf{0.344} & 0.312 & 0.213\tabularnewline
 &  & Number of Pronouns & BOI FP & 0.341 & \textbf{0.364} & \textbf{0.252}\tabularnewline
\cmidrule{2-7} 
& \multirow{2}{*}{HF} & Number of Lists & \multirow{2}{*}{None} & \textbf{0.114} & 0.021 & 0.015\tabularnewline
 &  & Number of P Tags &  & 0.110 & \textbf{0.123} & \textbf{0.084}\tabularnewline
\cmidrule{2-7} 
 & \multirow{2}{*}{WFF} & Mean Rank Medical Reddit & BOI NFP & \textbf{0.387} & 0.312 & 0.214\tabularnewline
 &  & 50th percentil Medical Reddit & JST NFP & 0.351 & \textbf{0.315} & \textbf{0.216}\tabularnewline
\cmidrule{2-7} 
 & \multirow{2}{*}{MLR} & Neural Network Regressor & JST NFP & \textbf{0.454} & \textbf{0.373} & 0.258\tabularnewline
 &  & Random Forest Regressor & BOI NFP & 0.389 & 0.355 & \textbf{0.264}\tabularnewline
\cmidrule{2-7} 
 & MLC & Multinomial Naive Bayes & JST FP & \textbf{0.461} & \textbf{0.391} & \textbf{0.318}\tabularnewline
\bottomrule
\end{tabular}
} %%%%% ---- 
\end{table}


Using the CLEF eHealth 2015 and 2016 collections, we studied the correlations of methods to estimate Web page understandability (Table~\ref{tab:doc_features}), compared with human assessments. For each category of understandability estimation, Table~\ref{tab:top_corr_metrics} reports the methods with highest Pearson, Spearman or Kendall correlations. For each method, we used the best preprocessing settings; a study of the impact of preprocessing is reported in Section~\ref{sec:which_preprocessing}.

Overall, Spearman and Kendall correlations obtained similar results (in terms of which methods exhibited the highest correlations): this was expected as, unlike Pearson, they are both rank-based correlations.

%For surface level readability measures, SMOG had the highest correlations for CLEF 2015 and DCI for CLEF 2016, regardless of correlation measure. These results resonated with those obtained for the category of raw components of readability formulas. In fact, the polysyllable words measure, which is the main feature used in SMOG, had the highest correlation for CLEF 2015 among these methods. Similarly, the number of difficult words, which is the main feature used in DCI, had the highest correlation for CLEF 2016 among these methods.
For traditional readability measures, SMOG had the highest correlations for CLEF 2015 and DCI for CLEF 2016, regardless of correlation measure. These results resonated with those obtained for the category of raw components of readability formulas. %In fact, the polysyllable words measure and the number of difficult words are, respectively, part of the SMOG and DCI formulas, which had the highest correlation for CLEF 2015 and 2016 among these methods.
In fact, the polysyllable words measure, which is the main feature used in SMOG, had the highest correlation for CLEF 2015 among these methods. Similarly, the number of difficult words, which is the main feature used in DCI, had the highest correlation for CLEF 2016 among these methods.

When examining the expert vocabulary category, we found that the number of MeSH concepts obtained the highest correlations with human assessments; however its correlations were significantly lower than those achieved by the best method from the consumer medical vocabularies category, i.e. the scores of CHV concepts. For the natural language category, we found that the number of pronouns, the number of stop words and the number of out of vocabulary words had the highest correlations -- and these were even higher than those obtained with MeSH and CHV based methods. In turn, the methods that obtained the highest correlations among the HTML category (counts of P tags and list tags) exhibited overall the lowest correlations compared to methods in the other categories. P tags are used to create paragraphs in a Web page, being thus a rough proxy for text length. 
Among methods in the word frequency category, the use of Medical Reddit (but also of PubMed) showed the highest correlations, and these were comparable to those obtained by the readability formulas. 

Finally, regressors and classifiers exhibited the highest correlations amongst all methods: in this  category, the  eXtreme Gradient Boosting (XGB) regressor and the multinomial Naive Bayes best correlated with human assessments. 

