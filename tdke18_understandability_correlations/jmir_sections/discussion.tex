%\vspace{-10pt}

\section*{Discussion}

%\todo{Need to move the discussion part of each method here...}

%\subsection*{Conclusion}
%\label{sec:conclusion_doc_analysis}



%The empirical experiments suggested that: (1) machine learning methods based on regression are best suited to estimate the understandability of health Web pages; (2) preprocessing does affect effectiveness (both for understandability prediction and document retrieval), although, compared to other methods, ML-based methods for understandability estimation are less subject to variability due to poor preprocessing; (3) learning to rank methods can be specifically trained to promote more understandable search results. 

\subsection*{Principal Findings}

The empirical experiments suggested that:

\vspace{-4pt}
\begin{enumerate}[leftmargin=*]
	\item Machine learning methods based on regression are best suited to estimate the understandability of health Web pages;
	\item Preprocessing does affect effectiveness (both for understandability prediction and document retrieval), although, compared to other methods, ML-based methods for understandability estimation are less subject to variability caused by poor preprocessing;
	\item Learning to rank methods can be specifically trained to promote more understandable search results, while still providing an effective trade-off with topical relevance.
\end{enumerate} 


\subsection*{Limitations}
In this work, we relied on data collected through the CLEF 2015 and CLEF 2016 evaluation efforts to evaluate the effectiveness of methods that estimate the understandability of the Web pages. These assessments were obtained by asking ratings to medical experts and practitioners; although they were asked to estimate the understandability of the content as if they were the patients they treat, there may have been noise and imprecisions in the collection mechanism due to the subjectivity of the
task. Figure~\ref{fig:boxplot_corr_docs} highlights this by showing that the agreement between assessors is relatively low. A better setting may have been to directly recruit health consumers: the task would still have been subjective, but would have captured real ratings, rather than inferred or perceived ratings. Despite this, our previous work has shown that no substantial differences were found in the downstream evaluation of retrieval systems, when we acquired understandability assessments from health consumers for a subset of the CLEF 2015 collection~\cite{palotti16b}. 

Relevance assessments on the CLEF 2015 and 2016 collections are incomplete~\cite{clef15,clef16}, i.e. not all top ranked web pages retrieved by the investigated methods have an explicit relevance assessment. This is often the case in information retrieval, where the validity of experiments based on incomplete assessments has been thoroughly investigated~\cite{sanderson2010test}. Nonetheless, we carefully controlled for the impact unassessed documents had in our experiments by measuring their number and using measures like RBP that account for residuals and condensed variants. The residuals analysis has been reported in the appendix. 

\subsection*{Conclusions}

We have examined approaches to estimate the understandability of health Web pages, including the impact of HTML preprocessing techniques, and how to integrate these within retrieval methods to provide more understandable search results for people seeking health information. We found that machine learning methods are better suited than traditionally employed readability measures for assessing the understandability of health related web pages and that learning to rank is the most effective
strategy to integrate this into retrieval. We also found that HTML and text pre-processing do affect the effectiveness of both understandability estimations and of the retrieval process, although machine learning methods are less sensitive to this issue.

This article contributes to improving search engines tailored to consumer health search because it thoroughly investigates promises and pitfalls of understandability estimations and their integration into retrieval methods. The article further highlights which methods and settings should be used to provide better search results to health information seekers. As shown in Figure~\ref{fig:dist}, these methods would clearly improve current health-focused search engines. 
