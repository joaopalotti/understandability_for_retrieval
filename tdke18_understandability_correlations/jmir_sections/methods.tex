%\subsection{Data and Resources}
\section*{Methods}
\label{sec:data}

\subsection{Data Collection}

In this article, we investigated methods to estimate Web page understandability, including the effect HTML preprocessing pipelines and heuristics have, and their search effectiveness when employed within retrieval methods. To obtain both topical relevance and  understandability assessments, we used the data from the CLEF 2015 and 2016 eHealth collections. (Note that we refer to topical relevance simply as relevance in the reminder of the paper, when this does not cause confusion.)

The CLEF 2015 collection contains 50 queries and 1,437 documents that have been assessed relevant by clinical experts and have an assessment for understandability~\cite{clef15}. Documents in this collection are a selected crawl of health Web sites, of which the majority are certified HON Web sites.
The CLEF 2016 collection contains 300 queries and 3,298 relevant documents that also have been assessed with respect to understandability~\cite{clef16}. Documents in this collection belong to the ClueWeb12 B13 corpus, and thus are general English Web pages, not necessarily targeted to health topics, nor of a controlled quality (as are instead HON certified pages). 
Understandability assessments were provided on a 5-point Likert scale for CLEF 2015, and on a $[0,100]$ range for CLEF 2016 (0 indicates the highest understandability). 

To support the investigation of methods to automatically estimate the understandability of Web pages, 
%in Section~\ref{sec:which_preprocessing} (evaluation of preprocessing pipelines and heuristics), 
we further considered correlations between multiple human assessors (inter-assessor agreement). For CLEF 2015, we used the publicly available additional assessments made by unpaid medical students and health consumers collected by Palotti et al.~\cite{palotti16b} in a study of how medical expertise affects assessments. For CLEF 2016 we  collected understandability assessments for 100 documents. 
%Three members of our research team, which did not author this paper, were recruited to provide the assessments\footnote{The correlation of these additional assessments and CLEF's ground-truth is examined in Section~\ref{sec:which_preprocessing}}. 
Three members of our research team, which did not author this article, were recruited to provide the assessments (the correlation of these additional assessments and CLEF's ground-truth is examined further in this article).
The Relevation tool~\cite{koopman14} was used to assist with the assessments, mimicking the settings used in CLEF. %\mytodo{should we briefly show these results?}

\subsection{Evaluation Measures}

In the experiments, we used Pearson, Kendall and Spearman correlations to compare the understandability assessments of human assessors with estimations obtained by the considered approaches, under all combinations of pipelines and heuristics. Pearson correlation is used to calculate the strength of the linear relationship between two variables, while Kendall and Spearman measure the rank correlations between the variables. We opted to report all three correlation coefficients to allow for a thorough comparison to other work, as they are equally used in the literature. 

%For the retrieval experiments in Section~\ref{sec:results}, we used evaluation measures that use both relevance and understandability. 
For the retrieval experiments, we used evaluation measures that rely on both (topical) relevance and understandability. 
The uRBP measure~\cite{zuccon2016understandability} extends rank biased precision (RBP) to scenarios where multiple relevance dimensions are used. The measure is formulated as $uRBP(\rho) = (1 - \rho) \sum_{k=1}^{K} \rho^{k-1} r(d@k) u(d@k)$, where $r(d@k)$ is the gain for retrieving a relevant document at rank $k$ and $u(d@k)$ is the gain for retrieving a document of a certain understandability at rank $k$; $\rho$ is the RBP persistence parameter. This measure was an official evaluation measure used in CLEF (we also set $\rho=0.8$). 

A drawback of uRBP is that relevance and understandability are combined into a unique evaluation score, thus making it difficult to interpret whether improvements are due to more understandable or more topical documents being retrieved. To overcome this, we first separately calculated an RBP value for relevance and another for understandability, and then combined them into a unique effectiveness measure:

\begin{itemize}[leftmargin=*]
	\item $RBP_r@n(\rho)$: uses the relevance assessments for the top $n$ search results (i.e. this is the common RBP). We regarded a document as topically relevant if assessed as somewhat relevant or highly relevant.
	
    \item $RBP_u@n(\rho)$: uses the understandability assessments for the top $n$ search results. We regarded a document as understandable (1) for CLEF 2015 if assessed easy or somewhat easy to understand; (2) for CLEF 2016 if its assessed understandability score was smaller than a threshold $U$. We used $U = 40$, based on the distribution of understandability assessments. This distribution can be found in the online appendix.
	
    \item $H_{RBP}@n(\rho) = 2 \times \frac{RBP_r@n \times RBP_u@n}{RBP_r@n + RBP_u@n}$: combines the previous two RBP values into a unique measurement using the harmonic mean (in the same fashion that the $F_1$ measure combines recall and precision).
\end{itemize}

\noindent For all measures, we set $n=10$ because shallow pools were used in CLEF along with measures that focused on the top 10 search results (including $RBP_r@10$).

Along with these measures of search effectiveness, we also reported the number of unassessed documents, the RBP residuals,  $RBP^*_r@10$, $RBP^*_u@10$ and $H_{RBP}^*$, i.e. the corresponding measures calculated by ignoring unassessed documents. We did this to minimise pool bias since the pools built in CLEF were of limited size, and the investigated methods retrieved a substantial number of unassessed documents.


\subsection*{Understandability Estimators}
\label{sec:proxies}

%As reviewed in Section~\ref{sec:related}, 
Several methods have been used to estimate the understandability of health Web pages, with the most popular methods (at least in the biomedical literature) being readability formulas based on surface level characteristics of the text. Next, we outline the categories of methods to estimate understandability used in this work; an overview is shown in Table~\ref{tab:doc_features}. Some of these methods further expand measures used in the literature. 
 
\begin{table*}[tb]
\caption{Metrics used as understandability proxies; $\star$: raw values are used. $\diamondsuit$: values normalised by number of words in a documents are used. $\dagger$: values normalised by number of sentences in a document are used.}
\label{tab:doc_features}
\resizebox{1.\textwidth}{!}{
\begin{tabular}{llcll}
\cline{1-2} \cline{4-5} 
\textbf{Group} & \textbf{Metric} & \multirow{29}{*}{} & \textbf{Group} & \textbf{Metric}\tabularnewline
\cline{1-2} \cline{4-5} 
\multirow{8}{*}{\textbf{Traditional Readability Formulas}} & Automated Readability Index (ARI) \cite{ari67} &  & \multirow{26}{*}{\textbf{HTML Features}} & \# of Abbr tags\tabularnewline
 & Coleman-Liau Index (CLI) \cite{cli75} &  &  & \# of A tags\tabularnewline
 & Dale Chall Index (DCI) \cite{dale48} &  &  & \# of Blockquote tags\tabularnewline
 & Flesch-Kincaid Grade Level (FKGL) \cite{flesch75} &  &  & \# of Bold tags\tabularnewline
 & Flesch Reading Ease (FRE) \cite{flesch75} &  &  & \# of Cite tags\tabularnewline
 & Gunning Fog Index (GFI) \cite{gunning52} &  &  & \# of Div tags\tabularnewline
 & Lasbarhetsindex (LIX) \cite{lix} &  &  & \# of Forms tags\tabularnewline
 & Simple Measure of Gobbledygook (SMOG) \cite{smog69} &  &  & \# of H1 tags\tabularnewline
\cline{1-2} 
\multirow{10}{*}{\textbf{\makecell{\kern-2.8emRaw Components \\of Readability Measures}}} & \# of Characters $^{\star\diamondsuit\dagger}$ &  &  & \# of H2 tags\tabularnewline
 & \# of Words $^{\star\dagger}$ &  &  & \# of H3 tags\tabularnewline
 & \# of Sentences {$^{\star\diamondsuit}$} &  &  & \# of H4 tags\tabularnewline
 & \# of Difficult Words (Dale Chall list \cite{dale48})
$^{\star\diamondsuit\dagger}$ &  &  & \# of H5 tags\tabularnewline
 & \# of Words Longer than 4 chars $^{\star\diamondsuit\dagger}$ &  &  & \# of H6 tags\tabularnewline
 & \# of Words Longer than 6 chars $^{\star\diamondsuit\dagger}$ &  &  & \# of Hs (any H above)\tabularnewline
 & \# of Words Longer than 10 chars $^{\star\diamondsuit\dagger}$ &  &  & \# of Img tags\tabularnewline
 & \# of Words Longer than 13 chars $^{\star\diamondsuit\dagger}$ &  &  & \# of Input tags\tabularnewline
 & \# of Number of Syllables $^{\star\diamondsuit\dagger}$ &  &  & \# of Link tags\tabularnewline
 & \# of Polysyllable Words (>3 Syllables) $^{\star\diamondsuit\dagger}$ &  &  & \# of DL tags\tabularnewline
\cline{1-2} 
\multirow{6}{*}{\textbf{Medical Vocabularies}} & \# of Words with Medical Prefix $^{\star\diamondsuit\dagger}$ &  &  & \# of UL tags\tabularnewline
 & \# of Words with Medical Suffix $^{\star\diamondsuit\dagger}$ &  &  & \# of OL tags\tabularnewline
 & \# of Acronyms $^{\star\diamondsuit\dagger}$ &  &  & \# of List (DL + UL + OL)\tabularnewline
 & \# of ICD Concepts $^{\star\diamondsuit\dagger}$ &  &  & \# of Q tags\tabularnewline
 & \# of Drugbank $^{\star\diamondsuit\dagger}$ &  &  & \# of Scripts tags\tabularnewline
 & \# of Words in medical dict. (OpenMedSpel) $^{\star\diamondsuit\dagger}$ &  &  & \# of Spans tags\tabularnewline
\cline{1-2} 
    \multirow{6}{*}{\textbf{\makecell{\kern-2.5emConsumer Health\\ Vocabulary (CHV) \cite{zeng06} \\ \kern-6.2emFeatures}}} & CHV Mean Score for all Concepts $^{\star\diamondsuit\dagger}$ &  &  & \# of Table tags\tabularnewline
 & \# of CHV Concepts $^{\star\diamondsuit\dagger}$ &  &  & \# of P tags\tabularnewline
\cline{4-5} 
 & CHV Mean Score for Symptom Concepts $^{\star\diamondsuit\dagger}$ &  & \multirow{20}{*}{\textbf{Word Frequency}} & 25th percentil English Wikipedia\tabularnewline
 & \# of CHV Symptom Concepts $^{\star\diamondsuit\dagger}$ &  &  & 50th percentil English Wikipedia\tabularnewline
 & CHV Mean Score for Disease Concepts $^{\star\diamondsuit\dagger}$ &  &  & 75th percentil English Wikipedia\tabularnewline
 & \# of CHV Disease Concepts $^{\star\diamondsuit\dagger}$ &  &  & Mean Rank English Wikipedia\tabularnewline
\cline{1-2} 
\multirow{6}{*}{\textbf{\makecell{\kern-0.3emMedical Subject\\ Headers (MeSH)}}} & \# of MeSH Concepts $^{\star\diamondsuit\dagger}$ &  &  & Mean Rank English Wikipedia - Includes OV\tabularnewline
 & Average Tree of MeSH Concepts $^{\star\diamondsuit\dagger}$ &  &  & 25th percentil Medical Reddit\tabularnewline
 & \# of MeSH Symptom Concepts $^{\star\diamondsuit\dagger}$ &  &  & 50th percentil Medical Reddit\tabularnewline
 & Average Tree of MeSH Symptom Concepts $^{\star\diamondsuit\dagger}$ &  &  & 75th percentil Medical Reddit\tabularnewline
 & \# of MeSH Disease Concepts $^{\star\diamondsuit\dagger}$ &  &  & Mean Rank Medical Reddit\tabularnewline
 & Average Tree of MeSH Disease Concepts $^{\star\diamondsuit\dagger}$ &  &  & Mean Rank Medical Reddit ncludelude OV\tabularnewline
\cline{1-2} 
\multirow{20}{*}{\textbf{Natural Language}} & Positive Words $^{\star\diamondsuit\dagger}$ &  &  & 25th percentil Pubmed\tabularnewline
 & Negative Words $^{\star\diamondsuit\dagger}$ &  &  & 50th percentil Pubmed\tabularnewline
 & Neutral Words $^{\star\diamondsuit\dagger}$ &  &  & 75th percentil Pubmed\tabularnewline
 & \# of verbs $^{\star\diamondsuit\dagger}$ &  &  & Mean Rank Pubmed\tabularnewline
 & \# of nouns $^{\star\diamondsuit\dagger}$ &  &  &  Mean Rank Pubmed - Includes OV\tabularnewline
 & \# of pronouns $^{\star\diamondsuit\dagger}$ &  &  & 25th p. Wikipedia+Reddit+Pubmed  \tabularnewline
 & \# of adjectives $^{\star\diamondsuit\dagger}$ &  &  & 50th p. Wikipedia+Reddit+Pubmed \tabularnewline
 & \# of adverbs $^{\star\diamondsuit\dagger}$ &  &  & 75th p. Wikipedia+Reddit+Pubmed \tabularnewline
 & \# of adpositions $^{\star\diamondsuit\dagger}$ &  &  & Mean R. Wiki.+Reddit+Pubmed \tabularnewline 
 & \# of conjunctions $^{\star\diamondsuit\dagger}$ & &  & Mean R. Wiki.+Reddit+Pubmed - w. OV \tabularnewline
\cline{4-5} 
 & \# of determiners $^{\star\diamondsuit\dagger}$ & & \multirow{5}{*}{\textbf{Regressor}} & Linear Regressor\tabularnewline  
 & \# of cardinal numbers $^{\star\diamondsuit\dagger}$ &  &  & Gradient Boosting Regressor\tabularnewline
 & \# of particles or other function words $^{\star\diamondsuit\dagger}$ &  &  & Multi-layer Perceptron Regressor\tabularnewline
 & \# of other POS (foreign words, typos) $^{\star\diamondsuit\dagger}$ &  &  & Random Forest Regressor\tabularnewline
 & \# of punctuation $^{\star\diamondsuit\dagger}$ &  &  & Support Vector Machine Regressor\tabularnewline
\cline{4-5} 
 & Height of part-of-speech parser tree $^{\star\diamondsuit\dagger}$ &  &  \multirow{6}{*}{\textbf{Classifier}} & Logistic Regression\tabularnewline
 & \# of Entities $^{\star\diamondsuit\dagger}$ &  &  & Gradient Boosting Classifier\tabularnewline
 & \# of Stopwords $^{\star\diamondsuit\dagger}$ &  &  & Multinomial Naive Bayes\tabularnewline
 & \# of words not found in Aspell Eng. dict. $^{\star\diamondsuit\dagger}$ &  &  & Multi-layer Perceptron Classifier\tabularnewline
 &  &  &  & Random Forest Classifier\tabularnewline
 &  &  &  & Support Vector Machine Classifier\tabularnewline
\cline{1-2} \cline{4-5} 
\end{tabular}
}
\end{table*}


\textit{Traditional Readability Formulas (RF):}
%These include the readability formulas mentioned in Section~\ref{sec:related}, as well as other, less popular ones. A full list is provided in surveys by Collins-Thompson \cite{collins2014computational} and Dubay~\cite{dubay04}.
These include the most popular readability formulas~\cite{cli75,dale48,flesch75}, as well as other, less popular ones~\cite{ari67,gunning52,lix,smog69}. A full list is provided in surveys by Collins-Thompson \cite{collins2014computational} and Dubay~\cite{dubay04}.

\textit{Raw Components of Readability Formulas (CRF):}
These are formed by the ``building blocks'' used in the traditional readability formulas; examples of such building blocks include the average number of characters per word and the average number of syllables in a sentence. Words are divided into syllables using the Python package Pyphen~\cite{pyphen}.

\textit{General Medical Vocabularies (GMV):}
These include methods that count the number of words with a medical prefix or suffix, i.e. beginning or ending with Latin or Greek particles (e.g., amni-, angi-, algia-, arteri-), and text strings included in lists of acronyms or in medical vocabularies such as the International Statistical Classification of Diseases and Related Health Problems (ICD), Drugbank and the OpenMedSpel dictionary~\cite{openmedspel}. An acronym list from the ADAM database~\cite{zhou2006} was used. Methods in this category were matched with documents using simple keywords matching. An acronym list from the ADAM database~\cite{zhou2006} was used and methods in this group were matched with documents using simple keywords matching.

\textit{Consumer Medical Vocabulary (CMV):}
The popular MetaMap \cite{aronson10} tool was used to map the text content of Web pages to entries in  CHV~\cite{zeng06}.
We used the MetaMap semantic types to retain only concepts identified as symptoms or diseases. Similar approaches have been commonly used in the literature (e.g.,~\cite{pang16,agrafiotesA16,palotti16,yates13}).

\textit{Expert Medical Vocabulary (EMV):}
Similarly to the CHV features, we used MetaMap to convert the content of Web pages into MeSH entities, studying symptom and disease concepts separately. 

\textit{Natural Language Features (NLF):}
These included commonly used natural language heuristics such as the ratio of part-of-speech (POS) classes, the height of the POS parser tree, the number of entities in the text, 
the sentiment polarity and the ratio of words found in English vocabularies. The Python package NLTK~\cite{nltk} was employed for sentiment analysis, POS tagging and entity recognition. The GNU Aspell~\cite{aspell} dictionary was used as a standard English vocabulary and a stop word list was built by merging those of Indri~\cite{indri} and Terrier~\cite{terrier}. 
Discourse features, such as the distribution of POS classes and density of entity in a text, were previously studied in the task of understandability prediction~\cite{feng10} being superior to complex features such as entity co-reference and entity grid~\cite{barzilay08}. To the best of our knowledge, sentiment polarity was never investigated in this task. Our intuition is that the content produced by laypeople in patient forums or blogs (easy-to-read) is potentially more emotional than scientific publications (hard-to-read).

\textit{HTML Features (HF):}
These included the identification of a large number of HTML tags, which were extracted with the Python library BeautifulSoap~\cite{bs4}. The intuition for these features is that Web pages with many images and tables may explain and summarise health content better, thus providing more understandable content to the general public. 

\textit{Word Frequency Features (WFF):}
Generally speaking, common and known words are usually frequent words, while unknown and obscure words are generally rare. This idea  is implemented in readability formulas such as the DCI, which uses a list of common words and counts the number of words that fall outside this list (complex words)~\cite{dale48} and has shown success in other recent approaches~\cite{elhadad06,wu15}.
We extended these observations by studying corpus-wide word frequencies. 
We modelled word frequencies in a corpus in a straightforward manner: we sorted the word frequencies and normalised word rankings such that values close to 100 are attributed to common words and values close to 0 to rare words. Three corpora were analysed to extract word frequencies:

\begin{itemize}[leftmargin=*]
    \item \underline{Medical Reddit:} Reddit~\cite{reddit} is a Web forum with a sizeable user community which is responsible for generating and moderating its content. Any user can start a discussion or reply to a discussion. This forum is intensively used for health purposes: for example in the Reddit community AskDocs~\cite{redditaskdocs}, licensed nurses and doctors (subject to user identity verification) advise help seekers free of charge. We selected six of such communities
        (medical, AskDocs, AskDoctorSmeeee, Health, WomensHealth, Mens\_Health) and downloaded all user interactions available until September 1st 2017 using the Python library PRAW~\cite{redditapi}. In total 43,019 discussions were collected.

\item \underline{Medical English Wikipedia:} after obtaining a recent Wikipedia dump~\cite{wikipedia} (May 1st 2017), we filtered articles to only those containing an Infobox in which at least one of the following words appeared as a property: ICD10, ICD9, DiseasesDB, MeSH, MeSHID, MeshName, MeshNumber, GeneReviewsName, Orphanet, eMedicine, MedlinePlus, drug\_name, Drugs.com, DailyMedID, LOINC. A Wikipedia infobox is a structured template that appears on the right of Wikipedia pages summarising key aspects of articles.

%Figure~\ref{fig:hyperthermia} illustrates a Wikipedia page that is marked as medical because of its Infobox entries.
This process followed the method by Soldaini et al.~\cite{soldaini15}, which favours precision over recall when identifying a health-related article. This resulted in a collection of 11,868 articles. 
%Note that this procedure highly favors precision over recall. %\mytodo{In case we need space, I suggest we drop this figure}

\item \underline{PubMed Central:} PubMed Central~\cite{pubmed} is an online  database of biomedical literature. We used the collection distributed for the TREC 2014 and 2015 Clinical Decision Support Track~\cite{roberts16,trec15}, consisting of 733,191 articles. 
 
\end{itemize}

A summary of the statistics of these three collections is reported in Table~\ref{tab:collection_stats}. 
%We removed words that occurred less than 5 times in a corpus.  %% We include this detail if necessary in the final version of this paper
%Finally, 
Unless explicitly stated otherwise, we ignored out of vocabulary words in our feature calculations.

\begin{table}[!tb]
\centering    
\caption{Statistics for the collections used as background models for understandability estimations.}
\vspace{-0.3cm}
\label{tab:collection_stats}
\resizebox{0.4\textwidth}{!}{
\begin{tabular}{cccc}
\toprule 
\textbf{Statistic} & \textbf{Medical Wikipedia} & \textbf{Medical Reddit} & \textbf{PubMed Central}\tabularnewline
\midrule 
\textbf{Number of Docs.} & 11,868 & 43,019 & 733,191\tabularnewline
\textbf{Number of Words} & 10,655,572 & 11,978,447 & 144,024,976\tabularnewline
\textbf{Number of Unique Words} & 467,650 & 317,106 & 2,933,167\tabularnewline
\textbf{Avg. Words per Doc.} & 898.90 $\pm$ 1351.76 & 278.45 $\pm$ 359.70  & 227.22 $\pm$ 270.44 \tabularnewline
\textbf{Avg. Char per Doc.} & 5107.81 $\pm$ 7618.57  & 1258.44 $\pm$ 1659.96  & 1309.11 $\pm$ 1447.31 \tabularnewline
\textbf{Avg. Char per Word} & 5.68 $\pm$ 3.75  & 4.52 $\pm$ 3.52 &  5.76 $\pm$ 3.51 \tabularnewline
\bottomrule
\end{tabular}
} % End of resizebox
\vspace{-8pt}
\end{table}


\textit{Machine Learning on Text - Regressors (MLR) and Classifiers (MLC):} These include machine learning methods for estimating Web page understandability. While Collins-Thompson highlighted the promise of estimating understandability using machine learning methods, a challenge is identifying the background corpus to be used for training~\cite{collins2014computational}. To this aim, we used the three corpora detailed above, and assumed understandability labels according to the expected difficulty of documents in these collections:

\begin{itemize}[leftmargin=*]
    \item Medical Reddit (label 1): Documents in this collection are expected to be written in a colloquial style, and thus the easiest to understand. All the conversations are in fact explicitly directed to assist inexpert health consumers;
    \item Medical English Wikipedia (label 2): Documents in this collection are expected to be less formal than scientific articles, but more formal than a Web forum like Reddit, thus somewhat more difficult to understand;
    \item PubMed Central (label 3): Documents from this collection are expected to be written in a highly formal style, as the target audience are physicians and biomedical researchers.
\end{itemize}

Models were learnt using all documents from these collections after features were extracted using Latent Semantic Analysis (LSA) with 10 dimensions (this number of dimensions was chosen based on preliminary experiments with the Random Forest algorithm; we leave as future work a detailed study on the impact of different number of dimensions on other machine learning algorithms). We modelled a classification task as well as a regression task using these collections. Thus, after applying the same LSA transformation to test documents from CLEF, a continuous score was assigned to each document by a regressor, while each classifier assigned the documents to one of the three classes. %\footnote{In principle, regressors should output a continuous value between 1 and 3, but no restrictions are set and potentially any value can be assigned to a document.}

%Different labels for the regression could be employed, for example, a label 6 to PubMed Central documents would emphasize that these documents are explicitly made for expert users, being 3 times harder than Wikipedia ones. We did not explore the effects of different labels in this work, it is left as future work.



\subsection{Preprocessing Pipelines and Heuristics}
\label{sec:pipelines}

%As part of our study, we investigated the influence the preprocessing of Web pages has on the estimation of understandability, when this is estimated using the methods in \todo{Section~\ref{sec:proxies}}. 
As part of our study, we investigated the influence the preprocessing of Web pages has on the estimation of understandability when this is estimated using the methods described above.
We did so by comparing the combination of a number of preprocessing pipelines, heuristics, and understandability estimation methods with human assessments of Web page understandability. 
Our experiments extended those by Palotti et al.~\cite{palotti15} and provided a much thorough analysis, as they only evaluated surface level readability formulas and did not compare their results against human assessments. 
%\mytodo{NEW CONTRIBUTION FOUND...}

To extract the content of a Web page from the HTML source we tested: BeautifulSoup~\cite{bs4} (\textit{Naive}), which just naively removes HTML tags, Boilerpipe~\cite{kohlschutter10} (\textit{Boi}) and Justext~\cite{jan11} ({Jst}), which eliminates boilerplate text together with HTML tags. 
%Palotti et al.'s data analysis highlighted that the text in HTML tags often missed a correct punctuation mark and thus the text extracted from HTML fields like titles, menus, tables and lists could be interpreted as many short sentences or few very long sentences, depending on whether a period was forced at the end of fields/sentences. We thus implement the same two heuristics devised by them to deal with this: \textit{ForcePeriod (FP)} and \textit{DoNotForcePeriod (DNFP)}. The FP heuristic forces a period at the end of each extracted HTML field; while the DNFP does not. 
Palotti et al.'s data analysis highlighted that the text in HTML fields like titles, menus, tables and lists often missed a correct punctuation mark and thus the text extracted from them could be interpreted as many short sentences or few very long sentences, depending on whether a period was forced at the end of fields/sentences. We thus implemented the same two heuristics devised by Palotti et al. to deal with this: \textit{ForcePeriod (FP)} and \textit{DoNotForcePeriod (DNFP)}. The FP heuristic forces a period at the end of each extracted HTML field, while the DNFP does not. 


\subsection{Integrating Understandability into Retrieval}
\label{sec:method_ltr}

We then investigated how understandability estimations can be integrated into retrieval methods to increase the quality of search results. %We start by evaluating re-ranking search results purely basing on understandability estimations. 
Specifically, we considered three retrieval methods of differing quality for the initial retrieval. These included the best two runs submitted to each CLEF task, and a plain BM25 baseline (default Terrier parameters: $b=0.75$ and $k_1=1.2$). As understandability estimators we used the eXtreme Gradient Boosting (XGB) regressor~\cite{chen16}, as well as SMOG for CLEF 2015 and DCI for CLEF 2016. 
%These were the best performing approaches \todo{from Section~\ref{sec:beyond_readability}}. 
These were the selected as they were best performing readability formulas and machine learning method for each collection (details in the evaluation of understandability estimators in Results section).
Note that for XGB, for assessed documents we used 10-fold cross validation, training XGB on 90\% of the data, and used its predictions for the remaining 10\%. For unassessed documents, we trained XGB on all assessed data and applied this model to generate predictions. Different machine learning methods and feature selection schemes were experimented with; results are available in the online appendix. XGB was selected because its results were the best, although other methods followed similar trends.


To integrate understandability estimators into the retrieval process, we first investigated \textit{re-ranking} search results retrieved by the initial runs purely based on the understandability estimations. 
If all the search results from a run were to be considered, then such a re-ranking method may place at early ranks Web pages highly likely to be understandable, but possibly less likely to be topically relevant. To balance relevance and understandability, we only re-ranked the first $k$ documents. We explored rank cut-offs $k = 15, 20, 50$. Because evaluation was performed with respect to the first $n=10$ rank positions, the setting $k=15$ provided a conservative re-ranking of search results, while, $k=50$ provided a less conservative re-ranking approach.
%Results are presented in Section~\ref{results:reranking}.

As an alternative to the previous two-step ranking strategy for combining topical relevance and understandability, we explored the \textit{fusion} of two search results lists separately obtained for relevance and understandability. For this, we used the Reciprocal Rank Fusion (RRF) method~\cite{cormack09}, which was shown effective for combining two lists of search results based on their documents \textit{ranks}, rather than scores. This approach was selected above score-based fusion methods
because of the different scoring strategies and distributions employed when scoring for relevance compared to for understandability. For relevance, we used, separately, the three methods used for re-ranking (ECNU~\cite{song15} and KISTI~\cite{oh15} for CLEF2015, GUIR~\cite{soldaini16} and ECNU~\cite{song16} for CLEF 2016, and BM25 for both collections). For understandability, we used, separately, the estimations from SMOG/DCI and XGB. Also for this approach, we studied limiting the ranking of
results to be considered by the methods across the cut-offs $k=15, 20, 50$. % Results are presented in Section~\ref{results:fusion}.

\begin{table}[!t]
\centering
\caption{Learning to rank settings. }
\vspace{-0.3cm}
\label{tab:ltr}
\resizebox{0.35\textwidth}{!}{
\begin{tabular}{l|l|l}
\toprule 
Name & Feature Set & Labeling Function\tabularnewline
\midrule 
Combo 1 & IR features & F(R,U) = $R$\tabularnewline
Combo 2 & IR + Unders. features & F(R,U) = $R$ \tabularnewline
Combo 3 & IR + Unders. features & F(R,U) = $R \times (100 - U)/100$ \tabularnewline
Combo 4 & IR + Unders. features & F(R,U) = $\begin{cases} R & \text{if } U \le 40\\ 0 & \text{otherwise}  \end{cases}$ \tabularnewline
Combo 5 & IR + Unders. features & F(R,U) = $\begin{cases} 2 \times R & \text{if } U \le 40\\ R & \text{otherwise}  \end{cases}$\tabularnewline
\bottomrule
\end{tabular}
}% end resizebox
\vspace{-14pt}
\end{table}


Finally, we considered a third alternative to combine relevance and understandability: using \textit{learning to rank} with features derived from retrieval methods (IR features) and understandability estimators.
With the CLEF 2016 collection, we explored five combinations of label attribution and feature sets, maintaining the same pairwise learning to rank algorithm based on tree boosting (XGB).
These combinations are listed in Table~\ref{tab:ltr}, with $R$ being the relevance of documents and $U$ their understandability estimation. While the definitions of Combo 1 and 2 are straightforward, the other methods deserve some further explanation. In Combo 3, a penalty was proportionally assigned to documents according to how far their understandability score was from a target score $U$ (we used $U=40$). For example, a document with understandability 100 received no penalty, as 100 was the easiest level of understanding, while another with understandability 50 received a 50\% penalty, meaning that its relevance score was halved. Combo 4 and 5 were based on a fixed threshold applied to the understandability score: if the score was higher than the threshold $U=40$, then the original relevance score (for Combo 4) or a boosted value (for Combo 5) was assigned to the corresponding document.


\subsection{Additional Resources}
%\todo{Move this section to the discussion part?}
In order to keep this article as succinct as possible, we only reported a subset of the results. The remaining results (which show similar trends to those reported here) are made available in an online supplementary appendix for completeness: \url{https://sites.google.com/view/understandabilityontheweb/}. All data and code will be shared on GitHub upon acceptance.

